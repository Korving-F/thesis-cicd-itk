Testing often is an integral part of software development processes. When new code gets added to a project, old features and functionality need to be verified, code styles need to be enforced and security scans should reveal any vulnerabilities that may be present. This raises the overall code quality and security of the software under development.
Unfortunately these steps can be quite labor intensive and are therefore sometimes ignored or neglected. This is where CI (Continuous Integration) comes into play and together with CD (Continuous Delivery) will be generically referred to as continuous practices. See Section \ref{sec:continuous-practices} for a more in-depth description of and distinction between these concepts.

%\textit{Continuous Integration} refers to a development practice where developers frequently merge their code in a shared repository which sets off an automated build/test process and verifies the quality of the code submission, its integration with the existing code base and supporting services.\cite{fowler, thoughtworks-ci}
%The main aim of this practice is to automate good code development habits to catch and fix any code bugs as soon as they occur. This raises overall confidence in the code quality of the project.

%\textit{Continuous Delivery} is a term often used as an extension of and in conjunction with Continuous Integration. It adds to CI by dealing with application release automation, among others. This maintains a state where the application under development is always tested and ready to be deployed.\cite{continuous-delivery-2} These combined will be generically referred to as continuous practices.

Implementing continuous practices is done using automation servers which facilitate the build and test processes of the application. They are henceforth interchangeably referred to as either CI servers or CI/CD servers since they often are capable of automating tasks that range the scope of both concepts. Series of these automated tasks are generally referred to as a \textit{pipelines}.

Certidude is the software project under test and the subject of this work. It is an open-source CA (Certificate Authority) management tool designed to make signing and revocations of certificates easier.\cite{certidude} While Certidude currently already has a CI/CD solution in place (TravisCI) there are some issues that motivate the search for a replacement:
\begin{itemize}
    \item Since this CI/CD solution is cloud based, only a limited amount of resources are available and general control over the CI/CD server is minimal. Many of Certidude's integration tests rely on services that are all currently bootstrapped onto the same machine instead of on dedicated instances (e.g. Virtual Private Network Gateway (VPN), VPN Client, Samba as Domain Controller (DC) for Active Directory (AD) integration tests etc.). Implementing this has proven to be problematic for the current tool.
    \item Support for Ubuntu Xenial (16.04) VMs has only recently started.\cite{travis-os-support} Slow support for newer versions of Operating Systems (OS) could be an issue if the application and its tests can't be properly containerized since compatibility with deployment OS is important.
\end{itemize}

\pagebreak

A number of popular CI/CD server solutions are available, but all come with different features and are continuously being improved upon by their developers. It can be a time consuming task to sift through these solutions and determine which is best suited for the project at hand.

The goal of this work is to perform a comparative analysis of popular CI/CD server implementations and choose the one best suited for Certidude based on a series of determined use cases and requirements. This thesis contributes such an analysis. The aim is to have this tool of choice be implemented, configured and at least meet the minimum requirement of being able to replace the current tool. An additional aim of this work will be to explore and implement some common test and security utilities that can be used in CI/CD pipelines and which are currently not yet implemented. While deploying the code (CD) is considered out of scope, compiling OpenWrt images as a CI task will be explored as well.

A constraint on this work will be the number of CI/CD server implementations under analysis. There are dozens of candidate tools available which potentially could perform some of the needed tasks in one form or another. However, running tests in all of these would be too time consuming. This will also be shortly addressed during the analysis phase in Chapter \ref{chapter:analysis}.

Tests and a practical comparison were based on an application developed for this thesis and for which the various CI/CD servers were installed, configured and implementation-specific pipelines were created. These can be found in Appendices \hyperref[chapter:appendix-travis]{1}, \hyperref[chapter:appendix-jenkins]{2}, \hyperref[chapter:appendix-gitlab]{3}, \hyperref[chapter:appendix-tc]{4} and \hyperref[chapter:appendix-circleci]{5}. A recommendation was given to the main developer of Certidude, \supervisor, after which a choice between the tools was made.

This thesis contributes both a minimum working replacement for the current tool in use and its pipeline configurations as well as an expansion on functionality by the inclusion of multiple static code analysis tasks and an OpenWrt image creation task in the developed pipeline. These and supporting materials can be found in Appendices \hyperref[chapter:appendix-certidude-pipeline]{6}, \hyperref[chapter:appendix-sonar]{7}, \hyperref[chapter:appendix-vbox]{8} and \hyperref[chapter:appendix-openwrt]{9}. In addition a list of "future  work"  suggestions  was compiled in Section \ref{future-work}, which  can  be used  as part of  a  development road map for Certidude.

The author would like to thank the supervisors of this work, \cosupervisor{} and \supervisor, for their ideas, critical feedback and without whom this work would not exist. Many thanks go out as well towards TeamCity, Jenkins, Gitlab, CircleCI and Travis CI for providing access to their tools.