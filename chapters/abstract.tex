Certidude is an Estonia-based, open source project aiming to make VPN certificate 
management as easy as possible for everyday system administrators. Like many other software projects,
Certidude has a test-suite that should increase confidence in code quality and ensure the presence of desired
functionality when deployed. Certidude already tries to make use of development practices like 
Continuous Integration and Delivery to automate integration tests and other development best practices, 
but limitations of the current tool have dulled their usefulness for Certidude and created the need 
for a replacement.

During this research common CI/CD server components and integrations were explored and discussed,
a test application was developed and
a comparative analysis was performed between five popular CI/CD server implementations based on that
application. Gitlab CI revealed itself to be the best fit according to defined requirements so a replacement
pipeline for Certidude was implemented in that tool. 

The current test-suite of Certidude still needs a lot of work, so in addition a list of "\textit{future work}"
suggestions was compiled which can be used as a development road map. Additionally this thesis can be considered
a short guide into the world of CI/CD. 

See Chapters \ref{chapter:introduction} and \ref{chapter:concepts} for an overview of Continuous Integration and related concepts, Chapter \ref{chapter:analysis} for the comparative analysis and Chapter \ref{chapter:implementation} for the implementation. The developed test application and pipeline files can be found here: \url{https://github.com/Korving-F/thesis-cicd-examples}

The thesis is in English and contains 34 pages of text, 5 chapters, 20 figures, 2 tables.