Certidude on Eesti päritoluga, avatud lähtekoodiga projekt, mille
eesmärgiks on muuta igapäevane VPN-sertifikaatide haldamine 
süsteemiadministraatorite jaoks võimalikult lihtsaks. Nagu
mitmetel teistel tarkvaraprojektidel, on Certidude’il testimisprotseduur,
mis suurendab koodi kvaliteedikindlust ja tagab juurutamisel soovitud
funktsionaalsuse. 

Certidude'i sihiks on kasutada arenduses pidevintegratsiooni  ja
-valmidust (CI/CD), et automatiseerida integratsiooniteste ja teisi
tunnustatud arendustavasid. Praeguse töövahendi piirangud on siiski
vähendanud nende kasulikkust Certidude’i jaoks ja tinginud selle
asendamise vajaduse.

Käesolevas lõputöös uuritakse levinumaid CI/CD serveri komponente ja
lahendusi, töötatakse välja testimisrakendus ja võrreldakse selle abil
viit levinud CI/CD serverilahendust. Gitlab CI vastas defineeritud
nõuetele kõige paremini, mistõttu Certidude’is võeti kasutusele see
töövahend.

Olemasolev Certidude’i testimisprotseduur vajab veel palju täiustamist,
mistõttu on koostatud ettepanekute nimekiri, mida saab kasutada arenduse
tegevuskavana. Lisaks võib seda lõpu tööd käsitleda lühikese juhendina
CI/CD maailma. Pidevintegratsioonist ja sellega seotud mõistetest
antakse ülevaade peatükkides \ref{chapter:introduction} ja \ref{chapter:concepts}, peatükis \ref{chapter:analysis} asub võrdlev analüüs ja
peatükis \ref{chapter:implementation} realisatsioon. Arenduse lisafailid võib leida  aadressilt 
\url{https://github.com/Korving-F/thesis-cicd-examples}.

Lõputöö on kirjutatud inglise keeles ning sisaldab teksti 34 leheküljel, 5 peatükki, 20 joonist, 2 tabelit.