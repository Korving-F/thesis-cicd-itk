When developers write software and want to properly test, build and deploy their code they can spend a lot of time on manual tasks to do so. Continuous Integration and Continuous Delivery practices aim to automate these best practices through the use of automation servers. This should ensure code is tested as much as possible while reducing the amount of time spent on performing manual tasks as well as reduce the number of mistakes made during execution. They enforce that these steps are actually performed and increase confidence in the secure and proper functionality of the code itself.

This thesis compared and analyzed various popular CI/CD servers, integrations and their suitability for Certidude to see what solution could replace the one currently in use. This was also done to relieve fundamental complaints of the current tool, continue to be able to automatically run tests as well as improve and expand on the defined pipeline. These alternative solutions were explored, tested and compared by using a dummy application developed by the author for this thesis and writing pipeline definitions for each one of these tools. Gitlab CI was found to be the most relevant CI/CD server to Certidude by both the comparative analysis as by the main developer of Certidude, \supervisor.

This thesis contributed a working replacement installation and pipeline for the current CI/CD setup in use as well as an expansion on functionality by the inclusion of multiple static code analysis tasks and an OpenWrt image creation task in the developed pipeline. A list of "future work" suggestions was compiled which can be used as part of a development road map for Certidude.